% ------------------------------------------------
%          FILE:  salto-quantico.tex
%       CREATED:  Dom 30/Dez/2012 hs 15:58
%   LAST CHANGE:  2013 Mar 27 08:35:47 PM
%        AUTHOR:  Sérgio Luiz Araújo Silva
%          SITE:  http://vivaotux.blogspot.com
%       TWITTER:  @voyeg3r
%         SKYPE:  sergioaraujosilva
% -------------------------------------------------

\chapter{O salto quântico do aprendizado}\label{salto-quantico}
\index{Salto quântico}

\section{Nem sempre o difícil é o certo}\label{sec:abelha}
\index{Fluencia}

{\footnotesize from: Fluência em inglês \dots -- \href{http://goo.gl/CcuIk}{Fluency Academy no Facebook}\footnote{http://goo.gl/CcuIk}}

\dots Eu estava na sala de estar de minha casa. É um lugar pacífico com visão
para o lago Kootenay na pequena montanha de Nelson, British Columbia, Canadá.
Passava de meio dia, no final de junho, e eu estava testemunhando uma luta de
vida e morte a apenas alguns passos de distancia. Na sala, comigo, estava uma
grande abelha, que estava presa na casa, e tentando sair, ela estava usando
todas as suas forças tentando voar através do vidro janela da sala de estar.
Repetidas vezes, suas asas zumbiram intensamente, a abelha desesperada se
jogava no vidro da janela. A estratégia da abelha para escapar era fácil de
perceber -- ``tentar com força'' (try harder). Claramente esta estratégia não
estava funcionando! No outro lado da sala, dez passos de distancia, a porta de
traz da casa estava estava aberta, dez segundos de voo e a ``embaraçada
abelha'' poderia alcançar a liberdade. Com apenas um pouco de esforço, e uma
pequena parcela de coragem de voar através do ``desconhecido'', a abelha
poderia facilmente se libertar alcançando seu objetivo que era ir para o lado
de fora. A possibilidade de descoberta está lá. Uma simples mudança em seu modo
de agir irá trazer o resultado desejado, isto pode ser fácil! Porque a abelha
não tentou outro método? Porque ela continuou com seu método quando claramente
ele não estava funcionando? Muitas vezes nós todos ficamos ofuscados ``tentando
duro'', com métodos que não parecem funcionar. Isto parece ser especialmente
verdade quando se trata de nossa habilidade no idioma inglês.

Mas ``tentando duro'' para desenvolver a habilidade de se comunicar
e a confiança não é necessariamente o melhor método para alcançar seu objetivo
de falar com confiança o inglês fluente. Algumas vezes, de fato, a estratégia
do ``Trabalho duro'' é em grande parte do problema que impede você de alcançar
seu objetivo.

\section{O que é um salto quântico?}
\index{Salto quântico!O que é salto quântico}
% fazer referência ao conteúdo quantum leap
O salto quântico é uma palavra tomada do vocabulário da nova Física - A Física
Quântica é a ciência que ajudou-nos a ter a televisão, computadores,
comunicação via satélite e energia nuclear. A Física Quântica tem sido descrita
como a mais poderosa ciência que a humanidade jamais descobriu. E ela requer
que nós repensemos as ideias que nós temos sobre tempo e espaço, e como
a consciência humana trabalha.

A Física Quântica também tem coisas incríveis para dizer sobre você, seu
potencial, e o poder da mente. Para simplificar, como seres humanos nós teremos
que reexaminar nossas ideias de como o universo trabalha e como nós nos
encaixamos nele. {\em Margaret J. Wheatley}, em seu premiado livro, ``Liderança
e a Nova Ciência'', descreve um salto quântico como:

\begin{quotation}
\noindent
``\dots Quando um elétron pula de uma órbita para outra sem passar através de estágios
intermediários. Está em um lugar e repentinamente em outro.''
\end{quotation}

Os Físicos que estudam o a ciência quântica tem notado que os elétrons fazem
estes ``saltos'' de um nível para outro nível vários estágios acima \dots\ sem
nenhum real esforço, e sem ir passo-a-passo do ponto de partida para
o ponto de chegada. O que está acontecendo aqui? Porquê isto acontece?
E é possível para você como um indivíduo dar um `salto' tão grande e sem esforço em
suas habilidades de comunicação em inglês e nível de confiança? É possível para você fazer
o Salto Quântico no Inglês?

\newpage
\section{A estratégia do Salto Quântico}\label{sec:estrategia}
\index{Salto quântico!Estratégia}

A estratégia do salto quântico, segundo Daniel E. Cotton baseia-se
na ideia de que você como indivíduo pode experimentar um salto quântico
no aprendizado, ou seja, pode alcançar um nível vários estágios acima, para
isso ele desenvolveu o que chama fluency framework, o mesmo tem alguns princípios:

\vspace{0.3\baselineskip}
\noindent {\bf As armadilhas do aprendizado} \\
Existem seis armadilhas que lhe impedem de chegar a sua fluência:
\begin{description}
	\item [Be reasonable]: Ser moderado em suas pretensões e expectativas, ou seja pensar pequeno.
		\item [More the same]: o método que lhe trouxe até este nível foi útil mas ele não lhe levará aos próximos níveis,
			isto se baseia na afirmação de Einstein:\\ {\footnotesize \ding{42} ``Insanity: doing the same thing over and over again and expecting different results'' }.
		\item [Play it safe]: O medo de correr riscos, especialmente de cometer erros ou de tentar algo novo como
			no caso da abelha na seção \ref{sec:abelha} página~\pageref{sec:abelha}.
		\item [Do it alone]: Baseia-se no livro ``O Segredo'', em que o universo conspira a seu favor, por exemplo:
			dizendo para muitas pessoas que está estudando inglês você cria compromisso.
		\item [Disengaged heart]: Se você acredita ser capaz você busca e acha os meios, se você não se acha capaz
			você não tem o mesmo entusiasmo e para no meio do caminho.
		\item [ Need more preparation]: Segundo Daniel E. Cotton a maioria das pessoas já possuiu
			o vocabulário necessário para ser fluente, veja a seção {\em Completando frases} na página \pageref{sub:frases}.
\end{description}

\noindent
Também segundo Daniel E. Cotton {\em Phrasal Verbs, Idioms, Linking Words, vocabulário etc.} fazem parte do que ele chama
de parte mecânica do aprendizado, a parte psicológica involve os pontos citados nas armadilhas do aprendizado, devemos concentrar
80\% do esforço na parte psicológica e apenas 20\% na parte mecânica.

\vspace{0.3\baselineskip}
{\footnotesize \ding{42} ``The reasonable man adapts himself to the world; the unreasonable one persists
in trying to adapt the world to himself.
Therefore, all progress depends on the unreasonable man.'' \\
                                 -- George Bernard Shaw}

\subsection{Completando frases}\label{sub:frases}

Neste ponto, muitos estudantes de inglês com os quais falo dirão: ``Este é um
grande método Daniel\footnote{Daniel E. Cotton}, mas eu não sei como fazer.''
Para estes estudantes, eu respondo gentilmente, ``Sim, na verdade você sabe.
Todas as respostas que você necessita estão na sua mente. Você necessita apenas
fazer as perguntas certas.'' Então, para provar-lhes isto, eu frequentemente
lhes dou uma atividade de complementação de sentenças, eu escrevo ou digo
o começo de uma sentença, e os deixo finaliza-la. Por exemplo, eu inicia
a seguinte sentença: \vspace{0.3\baselineskip}

``Somo things I can do now to improve my English communication skills are \dots''
\vspace{0.3\baselineskip}

É incrível o quão rápido eles conseguem completar as sentenças. Eles imediatamente dirão \dots

\setlist[1]{itemsep=-4pt}
\begin{itemize}
		\item Take an English speaking colleague out to lunch
		\item Find an online English instructor to work with
        \item Model great English speakers and presenters
        \item Form an English Master Mind group with like-minded professionals
        \item etc \dots
\end{itemize}

A maioria dos estudantes ficam surpresos em o quão rapidamente eles podem
completar a atividade de complementação de sentença. Eles sentem um novo senso
de confiança quando se dão conta de que já tem a respostas dentro deles.

Eu vou dar a você atividades de complementação de sentenças neste livro, eu
os chamo de {\em fundamentos de fluência}. Finalize cada uma das sentenças com
várias respostas. Seja honesto o quanto possível
sem parar para pensar muito sobre a mesma. Não se preocupe se o final faz ou
não sentido, se é profunda ou mesmo se é verdadeira. Simplesmente escreva o que
vier a sua mente \dots \hspace{0.3\baselineskip} \dots  mas escreva algo.

When you complete these simple yet profound statements, you will start to
immediately experience a shift in your thinking and actions. You will
experience a Quantum English Leap!

Quando você completar estas simples mas profundas afirmações, você imediatamente
começará a experimentar uma mudança em seu modo de pensar e em suas ações. Você
experimentará o salto quântico no inglês! Você irá experimentar o salto quântico
do aprendizado do inglês.

\vspace{0.3\baselineskip}
\noindent
Complete the following sentence using only {\bf one} or {\bf two} words:\\
Fluency is \dots

\vspace{0.3\baselineskip}

\noindent
Fill in the blank: \newline
Yesterday, I read \dots ? pages in English.

\vspace{0.3\baselineskip}
\noindent
Complete the sentences below: \\
1. One thing about my English that I am {\bf proud} about right now is \dots \newline
2. One thing about my English that I am {\bf grateful} for right now is \dots \newline
3. One thing about my English that makes me {\em excited} right now is \dots \newline
4. One thing about my English that I am {\bf committed} to right now is \dots \newline
\newline
Question: \newline
\noindent
What do {\bf you} do to raise your `competence' in English? \newline
What do {\bf you} you do become a more `proficient' English speaker? \newline
What do {\bf you} you to do to speak English more correctly? More accurately? \newline

\noindent É interessante notar que a proposta do ``salto quântico do
aprendizado'' engloba a ideia de {\em não estudar palavras isoladas} vista na seção
\ref{sub:aprenda-contextualmente} página \pageref{sub:aprenda-contextualmente},
assim como a ideia dos {\em blocos sonoros} vista na seção \ref{sec:blocos-sonoros}
na página \pageref{sec:blocos-sonoros}.

\begin{multicols}{2}
{\footnotesize
``Every hour I’m [learning a language] feels like a minute. Every minute I am
away from [the language I'm learning] feels like an hour.''

``\dots\ And that is the secret to how to become a polyglot in minutes, not
hours, months, or years. It’s to absolutely love it, so that studying isn’t
a chore; it isn’t a task you want to get out of the way so that you can reach
that fluency you lust for. No, lust fizzles – but if you love the language, if
you love the language-learning process, those hours, those months, and those
years, they’ll fly by.''

\vspace{0.3\baselineskip}
\noindent
The above quote is from (jump to 8’10” in the video) and this post was inspired
by Anthony Lauder from FluentCzech‘s YouTube video entitled Become a Polyglot
in Minutes not Years, which you can view here: \href{http://goo.gl/IB7Hf}{http://goo.gl/IB7Hf} on youtube

\noindent
source: \href{http://mandarinfromscratch.com/2011/06/}{link.}}
\vfill \columnbreak

{\footnotesize
``Toda hora que eu estou [aprendizagem de uma língua] parece um minuto. Cada
minuto que eu estou longe de [a língua que estou aprendendo] parece uma hora.''

``\dots\  E esse é o segredo de como se tornar um poliglota em minutos, não em
horas, meses ou anos. É absolutamente amor, de modo que o estudo não é uma
tarefa, não é uma tarefa que você quer sair do caminho para que você possa
chegar a essa fluência que você ansiar. Não, luxúria fizzles - mas se você ama
a língua, se você ama o processo de aprendizagem da língua, as horas, os meses
e os anos, eles vão voar.''

\vspace{0.3\baselineskip}
\noindent
A citação acima é de (ir para 8'10 "no vídeo) e este post foi inspirado
por Anthony Lauder do vídeo no YouTube intitulado FluentCzech Torne-se um poliglota
em minutos, não anos, que você pode ver aqui: \href{http://goo.gl/IB7Hf}{no youtube}

\noindent
fonte: \href{http://mandarinfromscratch.com/2011/06/}{link}}
\end{multicols}
